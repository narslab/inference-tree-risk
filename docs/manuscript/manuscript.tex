\documentclass[11pt,twoside]{article}
\usepackage{etex, soul}
\newcommand{\num}{6{} }

\raggedbottom

%geometry (sets margin) and other useful packages
\usepackage{geometry}
\geometry{top=1.2in, left=1in,right=1in,bottom=1in,headsep=6pt}
 \usepackage{graphicx,booktabs,calc}

%=== GRAPHICS PATH ===========
\graphicspath{{./images/}}
% Marginpar width
%Marginpar width
\newcommand{\pts}[1]{\marginpar{ \small\hspace{0pt} \textit{[#1]} } }
\setlength{\marginparwidth}{.5in}
%\reversemarginpar
%\setlength{\marginparsep}{.02in}

%% Fonts
% \usepackage{fourier}
% \usepackage[T1]{pbsi}

\usepackage{lmodern}
\usepackage[T1]{fontenc}
%\usepackage{minted}

\usepackage{rotating}
%% Cite Title
\usepackage[style=numeric,backend=biber,natbib,sorting=none,maxcitenames=2,maxbibnames=99,doi=false,isbn=false,url=false,eprint=false]{biblatex}
\addbibresource{../ai-trees-references.bib}

%%% Counters
\usepackage{chngcntr,mathtools}
\counterwithout{figure}{section}
\counterwithout{table}{section}

\numberwithin{equation}{section}

%% Captions
\usepackage{caption}
\captionsetup{
  labelsep=quad,
  justification=raggedright,
  labelfont=sc
}

%AMS-TeX packages
\usepackage{amssymb,amsmath,amsthm}
\usepackage{bm}
\usepackage[mathscr]{eucal}
\usepackage{colortbl}
\usepackage{color}

\DeclareRobustCommand{\hlcyan}[1]{{\sethlcolor{cyan}\hl{#1}}}

\usepackage{epstopdf,subfigure,hyperref,enumerate,polynom,polynomial}
\usepackage{multirow,minitoc,fancybox,array,multicol}

\definecolor{slblue}{rgb}{0,.3,.62}
\hypersetup{
    colorlinks,%
    citecolor=blue,%
    filecolor=blue,%
    linkcolor=blue,
    urlcolor=slblue
}

%%%TIKZ
\usepackage{tikz}
\usepackage{pgfplots}
\usepackage{pgfplotstable}
\usepackage{pgfgantt}
\pgfplotsset{compat=newest}

\usetikzlibrary{arrows,shapes,positioning}
\usetikzlibrary{decorations.markings}
\usetikzlibrary{shadows,automata}
\usetikzlibrary{patterns,fit}
%\usetikzlibrary{circuits.ee.IEC}
\usetikzlibrary{decorations.text}
% For Sagnac Picture
\usetikzlibrary{%
    decorations.pathreplacing,%
    decorations.pathmorphing%
}

%
%Redefining sections as problems
%
\makeatletter
\newenvironment{question}{\@startsection
	{section}
	{1}
	{-.2em}
	{-3.5ex plus -1ex minus -.2ex}
    	{1.3ex plus .2ex}
    	{\pagebreak[3]%forces pagebreak when space is small; use \eject for better results
	\large\bf\noindent{Question }
	}
	}
	%{\vspace{1ex}\begin{center} \rule{0.3\linewidth}{.3pt}\end{center}}
	%\begin{center}\large\bf \ldots\ldots\ldots\end{center}}
\makeatother

%
%Fancy-header package to modify header/page numbering
%
%\renewcommand{\chaptermark}[1]{ \markboth{#1}{} }
\renewcommand{\sectionmark}[1]{ \markright{#1}{} }

\usepackage{fancyhdr}
\pagestyle{fancy}
%\addtolength{\headwidth}{\marginparsep} %these change header-rule width
%\addtolength{\headwidth}{\marginparwidth}
%\fancyheadoffset{30pt}
%\fancyfootoffset{30pt}
\fancyhead[LO,RE]{\small  \it \nouppercase{\leftmark}}
\fancyhead[RO,LE]{\small Page \thepage}
\fancyfoot[RO,LE]{\small }% PR \num S-2015}
\fancyfoot[LO,RE]{\small }%\scshape MODL}
\cfoot{}
\renewcommand{\headrulewidth}{0.1pt}
\renewcommand{\footrulewidth}{0pt}
%\setlength\voffset{-0.25in}
%\setlength\textheight{648pt}


\usepackage{paralist}


%%% FORMAT PYTHON CODE
\usepackage{listings}
% Default fixed font does not support bold face
\DeclareFixedFont{\ttb}{T1}{txtt}{bx}{n}{8} % for bold
\DeclareFixedFont{\ttm}{T1}{txtt}{m}{n}{8}  % for normal

% Custom colors
\usepackage{color}
\definecolor{deepblue}{rgb}{0,0,0.5}
\definecolor{deepred}{rgb}{0.6,0,0}
\definecolor{deepgreen}{rgb}{0,0.5,0}

%\usepackage{listings}

% % Python style for highlighting
% \newcommand\pythonstyle{\lstset{
% language=Python,
% basicstyle=\footnotesize\ttm,
% otherkeywords={self},             % Add keywords here
% keywordstyle=\footnotesize\ttb\color{deepblue},
% emph={MyClass,__init__},          % Custom highlighting
% emphstyle=\footnotesize\ttb\color{deepred},    % Custom highlighting style
% stringstyle=\color{deepgreen},
% frame=tb,                         % Any extra options here
% showstringspaces=false            %
% }}

% % Python environment
% \lstnewenvironment{python}[1][]
% {
% \pythonstyle
% \lstset{#1}
% }
% {}

% % Python for external files
% \newcommand\pythonexternal[2][]{{
% \pythonstyle
% \lstinputlisting[#1]{#2}}}

% % Python for inline
% \newcommand\pythoninline[1]{{\pythonstyle\lstinline!#1!}}


\newcommand{\osn}{\oldstylenums}
\newcommand{\dg}{^{\circ}}
\newcommand{\lt}{\left}
\newcommand{\rt}{\right}
\newcommand{\pt}{\phantom}
\newcommand{\tf}{\therefore}
\newcommand{\?}{\stackrel{?}{=}}
\newcommand{\fr}{\frac}
\newcommand{\dfr}{\dfrac}
%\newcommand{\ul}{\underline}
\newcommand{\tn}{\tabularnewline}
\newcommand{\nl}{\newline}
\newcommand\relph[1]{\mathrel{\phantom{#1}}}
\newcommand{\cm}{\checkmark}
\newcommand{\ol}{\overline}
\newcommand{\rd}{\color{red}}
\newcommand{\bl}{\color{blue}}
\newcommand{\pl}{\color{purple}}
\newcommand{\og}{\color{orange!90!black}}
\newcommand{\gr}{\color{green!40!black}}
\newcommand{\nin}{\noindent}
\newcommand{\la}{\lambda}
\renewcommand{\th}{\theta}
\newcommand{\al}{\alpha}
\newcommand{\G}{\Gamma}
\newcommand*\circled[1]{\tikz[baseline=(char.base)]{
            \node[shape=circle,draw,thick,inner sep=1pt] (char) {\small #1};}}

\newcommand{\bc}{\begin{compactenum}[\quad--]}
\newcommand{\ec}{\end{compactenum}}

\newcommand{\p}{\partial}
\newcommand{\pd}[2]{\frac{\partial{#1}}{\partial{#2}}}
\newcommand{\dpd}[2]{\dfrac{\partial{#1}}{\partial{#2}}}
\newcommand{\pdd}[2]{\frac{\partial^2{#1}}{\partial{#2}^2}}


%%%%%%%%%%%%%%%%%%%%%%%%%%%%%%%%%%%%%%%%%%%%%%%%%%%
%%%%%%%%%%%%%%%%%%%%%%%%%%%%%%%%%%%%%%%%%%%%%%%%%%%

\begin{document}
%\vspace{-16ex}
\title{Predicting tree failure likelihood for utility risk mitigation via a novel convolutional neural network}
%\author{}
\date{}
\maketitle


\section{Introduction}
Utilities have been concerned with tree-related outages for many years, but much of the work to quantify them has been
done in-house: Utilities contract with a consultant who analyzes tree-related outages (or interruptions). In general,
tree-related outages are due to contact between tree parts and lines. Sometimes, the contact is due to growth of
branches into the lines; sometimes it is due to branch failure; sometimes it is due to tree failure (either uprooting or
trunk failure). When outages occur, the economic costs can be substantial. From 2005 to 2015, \cite{graziano2020wider}
computed the annual economic cost of outages in the state of Connecticut; the mean was approximately \$8.3 bn.  Outages
due to contact from branch growth or branch failure can be mitigated by line clearance tree trimming. Estimates vary,
but the annual amount that utilities spend on tree trimming operations is typically in the billions of dollars
\cite{guggenmoos2003effects}. Despite the comparatively high annual cost of tree trimming, studies have repeatedly shown
that trimming branches away from the wires reduces the number of outages (sometimes classified as “preventable”
outages). For example, in Massachusetts, \citet{simpson1996treecaused} reported that tree failure caused 40\% of
preventable tree-caused outages, but removing or severely pruning high risk trees improved reliability by 20\% to
30\%. In a more recent (and sophisticated) analysis in Connecticut, \citet{parent2019analysis} showed that enhanced tree
trimming reduced outages during storms. And using a statistical model to predict outages based on data from two Gulf
Coast states in the USA, \citet{nateghi2014power} demonstrated that prediction models were less uncertain when they
included the effect of tree trimming; models including wind speed without considering the effect of trimming were not as
accurate.

Trimming can be effective because both structurally-deficient and structurally-sound branches adjacent to or overhanging
the lines in proximity are removed, reducing the likelihood of contact.
% Trimming also reduces external loads on the tree (e.g., wind or snow and ice), which reduces the likelihood of
% whole-tree failure, whether by uprooting or stem breakage.  Although the annual cost of trimming is comparatively
% large, the economic cost of outages can far exceed it. A recent benefit cost analysis demonstrated a highly favorable
% benefit cost ratio for trimming relative to the economic costs of outages (Graziano et al. 2020).  Trimming,
However, it cannot eliminate tree-related outages. Failure of trees away from the right-of-way can still impact the
lines and cause outages \cite{guggenmoos2003effects}. The proportion of tree failures away from the wires that cause
outages varies and has not been rigorously quantified. \citet{guggenmoos2011treerelated} estimated that 95\% of
tree-caused outages in the Pacific Northwest region of the USA, were due to tree failure, and \citet{wismer2018targeted}
reported approximately 25\% of interruptions in Illinois, USA were caused by trees that uprooted or broke in the stem.
Predicting the likelihood of failure is an inexact science, but tree risk assessment best management practices have been
developed \cite{e.thomassmiley2017best,johnw.goodfellow2020best}. Assessing risk includes assessing the likelihood of
failure, the likelihood of impact, and the severity of consequences. Likelihood of failure depends on the anticipated
loads on the tree and its load-bearing capacity. The likelihood of impact depends on proximity to the target (the lines,
poles, and other hardware—“infrastructure”—in the case of utility tree risk assessment), the target’s occupancy rate
(which is constant for utility lines) and whether the target is sheltered, for example by neighboring trees. Severity of
consequences depends on the damage done to the infrastructure and, more importantly in some cases, the economic costs
and disruption associated with outages—this, in turn, is partially related to the size of the tree or tree part that
fails, and how much momentum it has when it impacts the infrastructure.

Assessing risk of individual trees can be costly because of the time required to complete the assessment. In some
situations, a less time-consuming assessment may be justified to reduce costs—a ``Level 1'' assessment (Smiley et
al. 2017). Studies have shown that trees with greater risk ratings were more likely to be detected from Level 1 risk
assessments conducted in a moving vehicle in Rhode Island, USA \cite{rooney2005reliability} and Florida, USA
\cite{koeser2016frequency} . The utility of Level 1 assessments in Rhode Island and Florida suggests that artificial
intelligence (AI) tools may be an effective way to reduce the cost of tree risk assessment.

AI-based image analysis is relatively widely used, even in engineering applications such as earthquake risk assessment
\cite{jiao2020artificial,salehi2018emerging} and structural health monitoring \cite{spencer2019advances}. A relevant
application for tree species identification using a convolutional neural network (CNN) was even recently demonstrated
\cite{fricker2019convolutional}. Yet, AI has not been applied to the problem of tree-utility line risk assessment---one that is
complicated by the very large number of tree species to be considered, seasonal variation in tree appearance and
associated risk and local meteorological conditions. However, the flexibility and power of CNNs, appears promising. In
this paper, we demons In the remainder of this paper, we first provide a background on CNN which is critical for
sustainability in critical infrastructural systems. Our goal is to further demonstrate an innovative automated approach
to tree risk assessment using an AI tool that can be readily deployed for use in various locations and also continually
improved through subsequent training on new datasets.

\section{Background}

 


\section{Data and Methods}
\subsection{Image data description}
Based on the  ISA’s Tree Risk Assessment Qualification (TRAQ) protocol, four categories of tree failure likelihood are defined. In this paper we only focus on three:
\begin{itemize}
\item \textbf{Probable}: failure expected under normal weather conditions within a given timeframe
\item \textbf{Possible}: failure expected under extreme weather conditions; but unlikely during normal weather conditions
\item \textbf{Improbable}: failure unlikely either during normal or extreme weather conditions 
\end{itemize}

\subsection{Data pre-processing}

\subsection{Convolutional neural network}

\section{Results and Discussion}


\section{Conclusion}

\printbibliography

\appendix


\end{document}

%%% Local Variables:
%%% mode: latex
%%% TeX-master: t
%%% End:
