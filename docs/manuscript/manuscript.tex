\documentclass[11pt,twoside]{article}
\usepackage{etex, soul}
\newcommand{\num}{6{} }

\raggedbottom

%geometry (sets margin) and other useful packages
\usepackage{geometry}
\geometry{top=1.2in, left=1in,right=1in,bottom=1in,headsep=6pt}
 \usepackage{graphicx,booktabs,calc}

%=== GRAPHICS PATH ===========
\graphicspath{{./images/}}
% Marginpar width
%Marginpar width
\newcommand{\pts}[1]{\marginpar{ \small\hspace{0pt} \textit{[#1]} } }
\setlength{\marginparwidth}{.5in}
%\reversemarginpar
%\setlength{\marginparsep}{.02in}

%% Fonts
% \usepackage{fourier}
% \usepackage[T1]{pbsi}

\usepackage{lmodern}
\usepackage[T1]{fontenc}
%\usepackage{minted}

\usepackage{rotating}
%% Cite Title
\usepackage[style=numeric,backend=biber,natbib,sorting=none,maxcitenames=2,maxbibnames=99,doi=false,isbn=false,url=false,eprint=false]{biblatex}
\addbibresource{../ai-trees-references.bib}

%%% Counters
\usepackage{chngcntr,mathtools}
\counterwithout{figure}{section}
\counterwithout{table}{section}

\numberwithin{equation}{section}

%% Captions
\usepackage{caption}
\captionsetup{
  labelsep=quad,
  justification=raggedright,
  labelfont=sc
}

%AMS-TeX packages
\usepackage{amssymb,amsmath,amsthm}
\usepackage{bm}
\usepackage[mathscr]{eucal}
\usepackage{colortbl}
\usepackage{color}

\DeclareRobustCommand{\hlcyan}[1]{{\sethlcolor{cyan}\hl{#1}}}

\usepackage{epstopdf,subfigure,hyperref,enumerate,polynom,polynomial}
\usepackage{multirow,minitoc,fancybox,array,multicol}

\definecolor{slblue}{rgb}{0,.3,.62}
\hypersetup{
    colorlinks,%
    citecolor=blue,%
    filecolor=blue,%
    linkcolor=blue,
    urlcolor=slblue
}

%%%TIKZ
\usepackage{tikz}
\usepackage{pgfplots}
\usepackage{pgfplotstable}
\usepackage{pgfgantt}
\pgfplotsset{compat=newest}

\usetikzlibrary{arrows,shapes,positioning}
\usetikzlibrary{decorations.markings}
\usetikzlibrary{shadows,automata}
\usetikzlibrary{patterns,fit}
%\usetikzlibrary{circuits.ee.IEC}
\usetikzlibrary{decorations.text}
% For Sagnac Picture
\usetikzlibrary{%
    decorations.pathreplacing,%
    decorations.pathmorphing%
}

%
%Redefining sections as problems
%
\makeatletter
\newenvironment{question}{\@startsection
	{section}
	{1}
	{-.2em}
	{-3.5ex plus -1ex minus -.2ex}
    	{1.3ex plus .2ex}
    	{\pagebreak[3]%forces pagebreak when space is small; use \eject for better results
	\large\bf\noindent{Question }
	}
	}
	%{\vspace{1ex}\begin{center} \rule{0.3\linewidth}{.3pt}\end{center}}
	%\begin{center}\large\bf \ldots\ldots\ldots\end{center}}
\makeatother

%
%Fancy-header package to modify header/page numbering
%
%\renewcommand{\chaptermark}[1]{ \markboth{#1}{} }
\renewcommand{\sectionmark}[1]{ \markright{#1}{} }

\usepackage{fancyhdr}
\pagestyle{fancy}
%\addtolength{\headwidth}{\marginparsep} %these change header-rule width
%\addtolength{\headwidth}{\marginparwidth}
%\fancyheadoffset{30pt}
%\fancyfootoffset{30pt}
\fancyhead[LO,RE]{\small  \it \nouppercase{\leftmark}}
\fancyhead[RO,LE]{\small Page \thepage}
\fancyfoot[RO,LE]{\small }% PR \num S-2015}
\fancyfoot[LO,RE]{\small }%\scshape MODL}
\cfoot{}
\renewcommand{\headrulewidth}{0.1pt}
\renewcommand{\footrulewidth}{0pt}
%\setlength\voffset{-0.25in}
%\setlength\textheight{648pt}


\usepackage{paralist}


%%% FORMAT PYTHON CODE
\usepackage{listings}
% Default fixed font does not support bold face
\DeclareFixedFont{\ttb}{T1}{txtt}{bx}{n}{8} % for bold
\DeclareFixedFont{\ttm}{T1}{txtt}{m}{n}{8}  % for normal

% Custom colors
\usepackage{color}
\definecolor{deepblue}{rgb}{0,0,0.5}
\definecolor{deepred}{rgb}{0.6,0,0}
\definecolor{deepgreen}{rgb}{0,0.5,0}

%\usepackage{listings}

% % Python style for highlighting
% \newcommand\pythonstyle{\lstset{
% language=Python,
% basicstyle=\footnotesize\ttm,
% otherkeywords={self},             % Add keywords here
% keywordstyle=\footnotesize\ttb\color{deepblue},
% emph={MyClass,__init__},          % Custom highlighting
% emphstyle=\footnotesize\ttb\color{deepred},    % Custom highlighting style
% stringstyle=\color{deepgreen},
% frame=tb,                         % Any extra options here
% showstringspaces=false            %
% }}

% % Python environment
% \lstnewenvironment{python}[1][]
% {
% \pythonstyle
% \lstset{#1}
% }
% {}

% % Python for external files
% \newcommand\pythonexternal[2][]{{
% \pythonstyle
% \lstinputlisting[#1]{#2}}}

% % Python for inline
% \newcommand\pythoninline[1]{{\pythonstyle\lstinline!#1!}}


\newcommand{\osn}{\oldstylenums}
\newcommand{\dg}{^{\circ}}
\newcommand{\lt}{\left}
\newcommand{\rt}{\right}
\newcommand{\pt}{\phantom}
\newcommand{\tf}{\therefore}
\newcommand{\?}{\stackrel{?}{=}}
\newcommand{\fr}{\frac}
\newcommand{\dfr}{\dfrac}
%\newcommand{\ul}{\underline}
\newcommand{\tn}{\tabularnewline}
\newcommand{\nl}{\newline}
\newcommand\relph[1]{\mathrel{\phantom{#1}}}
\newcommand{\cm}{\checkmark}
\newcommand{\ol}{\overline}
\newcommand{\rd}{\color{red}}
\newcommand{\bl}{\color{blue}}
\newcommand{\pl}{\color{purple}}
\newcommand{\og}{\color{orange!90!black}}
\newcommand{\gr}{\color{green!40!black}}
\newcommand{\nin}{\noindent}
\newcommand{\la}{\lambda}
\renewcommand{\th}{\theta}
\newcommand{\al}{\alpha}
\newcommand{\G}{\Gamma}
\newcommand*\circled[1]{\tikz[baseline=(char.base)]{
            \node[shape=circle,draw,thick,inner sep=1pt] (char) {\small #1};}}

\newcommand{\bc}{\begin{compactenum}[\quad--]}
\newcommand{\ec}{\end{compactenum}}

\newcommand{\p}{\partial}
\newcommand{\pd}[2]{\frac{\partial{#1}}{\partial{#2}}}
\newcommand{\dpd}[2]{\dfrac{\partial{#1}}{\partial{#2}}}
\newcommand{\pdd}[2]{\frac{\partial^2{#1}}{\partial{#2}^2}}


%%%%%%%%%%%%%%%%%%%%%%%%%%%%%%%%%%%%%%%%%%%%%%%%%%%
%%%%%%%%%%%%%%%%%%%%%%%%%%%%%%%%%%%%%%%%%%%%%%%%%%%

\begin{document}
%\vspace{-16ex}
\title{Predicting tree failure likelihood for utility risk mitigation via a novel convolutional neural network}
%\author{}
\date{}
\maketitle


\section{Introduction}
Power outages due to contact between tree parts and power lines annually cause tens of billions of dollars in economic and other disruptions throughout the United States, despite extensive efforts by utilities to mitigate and prevent those contacts.  Presently, the identification of potential contact between trees and power lines is labor intensive and time-consuming.  This paper describes an artificial intelligence and machine learning approach that automatically classifies trees, using only a single photograph and with a high degree of accuracy, into categories corresponding to probable failure, possible failure and improbable failure--categories corresponding to those used by utility arborists to assess risk of contact between trees and power lines.  This preliminary study demonstrates the possible efficacy of AI approaches to tree risk assessment and, following further development of the approach has the potential to reduce power outages and utility costs by allowing utilities to more effectively target their pruning and mitigation efforts.  

Contact between tree parts and power lines can take several forms: tree branches can grow into lines; branches can fail and fall onto lines; whole-tree failure can occur due to uprooting or trunk failure.  A study in the state of Connecticut provides some context for the amount of economic, documenting annual disruptions of \$8.3 billion between 2005 and 2015 \cite{graziano2020wider}.  That extremely high cost occurs despite extensive efforts on the part of utilities to mitigate such disruptions through active and aggressive pruning programs that, on their own cost billions of dollars annually \cite{guggenmoos2003effects}.  

Pruning, despite its high cost, has been found to be effective in reducing disruptions due to so-called 'preventable' contact incidents between trees and power lines.  For example, in Massachusetts, where tree failure was responsible for 40\% of preventable tree-caused outages, pruning was able to improve reliability by 20\% to 30\% \cite{simpson1996treecaused}, and those results were replicated in a study in Connecticut \cite{parent2019analysis}. The efficacy of pruning has also been shown in a study of two US Gulf Coast states that showed wind-induced power outage prediction models becoming less uncertain when pruning was included in the model \citet{nateghi2014power}.

Even effective pruning cannot, however, completely eliminate tree-related outages. Failure of trees away from the right-of-way can still impact the lines and cause outages \cite{guggenmoos2003effects}. The proportion of tree failures away from the wires that cause outages varies and has not been rigorously quantified. \citet{guggenmoos2011treerelated} estimated that 95\% of tree-caused outages in the Pacific Northwest region of the US, were due to tree failure, and \citet{wismer2018targeted} reported approximately 25\% of interruptions in Illinois, USA, were caused by trees that uprooted or broke in the stem. 


Predicting the likelihood of failure is an inexact science, but tree risk assessment best management practices have been developed \cite{e.thomassmiley2017best,johnw.goodfellow2020best}. Risk includes assessing the likelihood of failure, the likelihood of impact, and the severity of consequences. The likelihood of failure depends on the anticipated loads on the tree and its load-bearing capacity. The likelihood of impact depends on proximity to the target (the lines, poles, and other hardware---``infrastructure''---in the case of utility tree risk assessment), the target’s occupancy rate (which is constant for utility lines) and whether the target is sheltered, for example by neighboring trees. Severity of consequences depends on the damage done to the infrastructure and, more importantly in some cases, the economic costs and disruption associated with outages---this, in turn, is partially related to the size of the tree or tree part that fails, and how much momentum it has when it impacts the infrastructure.

Individual tree risk assessment can be costly because of the time it requires. In some situations, a less time-consuming assessment may be justified to reduce costs, i.e.\ a ``Level 1'' assessment (XXXSmiley et al. 2017). Studies have shown that trees with greater risk ratings were more likely to be detected from Level 1 risk assessments conducted in a moving vehicle in Rhode Island, USA \cite{rooney2005reliability} and Florida, USA \cite{koeser2016frequency} . The utility of Level 1 assessments in these areas suggests that artificial intelligence (AI) tools may be an effective way to reduce the cost of tree risk assessment.

The method described in the paper uses convolutional neural networks (CNN) to classify images of trees among three categories of failure likelihood: probable, possible and improbably.  The data used for training, testing and illustration of the method consists of XXX tree images that have been classified by the authors according to prevailing standards employed by professional utility arborists. 

The remainder of the paper provides background AI and risk assessment, utility tree risk assessment and CNNs. The goal is to further demonstrate an innovative automated approach to tree risk assessment using an AI tool that can be readily deployed for use in various locations and also continually improved through subsequent training on new datasets.



\section{Background}

%\subsection{AI and risk assessment}
AI-based image analysis is relatively widely used, even in engineering applications, such as earthquake risk assessment \cite{jiao2020artificial,salehi2018emerging} and structural health monitoring \cite{spencer2019advances}. Neural networks have been widely applied in the field of earthquake risk assessment (an excellent review is provided by 10.1177/8755293020919419 XXX), but the authors are not aware of attempts to operate directly on, for example, building images in the absence of technical structural data to predict seismic risk.  Neural networks have also been used to interrogate remote sensing data of the landscape to assess landslide risk ( 10.1007/s10346-020-01557-6 XXX)


A relevant application for tree species identification using a convolutional neural network (CNN) was even recently demonstrated \cite{fricker2019convolutional}. Yet, AI has not been applied to the problem of tree-utility line risk assessment---one that is complicated by the very large number of tree species to be considered, seasonal variation in tree appearance and associated risk and local meteorological conditions. 


%Neural networks developed over the past several decades from the introduction of the single-layer perceptron.
The groundbreaking study of \citet{hubel1959receptive} showed that visual perception in cats was a result of the
activation or inhibition of groups of cells in the visual cortex known as ``receptive fields.''  Further, they attempted
to map the cortical architecture in cats and monkeys \cite{hubel1962receptive,hubel1965receptive,hubel1968receptive}.
Subsequent attempts were then made to model neural networks that could be trained to automatically recognize visual patterns with modest performance \cite{rosenblatt1962principles,kabrisky1966proposed,giebel1971feature,fukushima1975cognitron}. However, the breakthrough came with ``neocognitron'' \cite{fukushima1980neocognitron}, which
was a self-learning neural network for pattern recognition that was robust to changes in position and shape distortion, a problem that plagued earlier efforts, including ``cognitron'' \cite{fukushima1975cognitron} proposed a few years earlier.

A few notable efforts demonstrated the neural networks for handwritten digit recognition
\cite{fukushima1988neocognitron,denker1988neural}, but these required significant preprocessing and feature
extraction. \cite{lecun1989handwritten} soon afterward introduced a multilayer neural network that mapped a feature in each neuron (representing a ``local receptive field'') via convolution. This network could also be trained by backpropagation like other existing neural networks and featured pooling operations for better distortion and
translation invariance. Further developments from this milestone yielded the LeNet-5 convolutional neural network which attained accuracy levels that rendered it commercially viable.

The big data revolution coupled with technological advancements that have made it possible to capture and store high resolution images have raised challenges that continue to be surmounted with successively high-performing
architectures. Over the past decade, some of these efforts resulted in significant breakthroughs in performance. AlexNet \cite{krizhevsky2012imageneta}, with 5 convolutional layers and 3 dense layers---one of the largest CNNs of its time, won the ILSVRC-2012\footnote{ImageNet Large Scale Visual Recognition Challenge; held annually from 2010 through 2017.} competition with a top-5 error rate of 15.3\% and served as a landmark in the Deep Learning subdomain. \citet{zeiler2014visualizing} then introduced ZFNet, besting the performance of AlexNet, and pioneered visualization techniques that were foundational for model inference and interpretability.  In the same year, GoogLeNet, a 22-layer network, was proposed \cite{szegedy2014going}, featuring the novel ``Inception module,'' which allowed for efficiency and accuracy in a very deep network. Subsequent improvements have been proposed to the original inception framework \cite{szegedy2015rethinking,szegedy2016inceptionv4}.  VGGNet \cite{simonyan2015very} also pushed the boundaries of depth with up 19 layers, achieving state-of-the-art performance at ILSVRC-2014. Finally, ResNet \cite{he2015deep} addressed the accuracy degradation problem that arises with increasing depth in a network by succesively fitting smaller sets of layers to the residual and employing skip connections. With these innovations, an unprecedented level of depth was achieved. Implementations with with 34, 50, 101 and 152 layers were demonstrated. ResNet-152 won first place in ILSVRC-2015.

Along with these developments in their architectures, CNNs have demonstrated viability for applications to image
classification, object and text detection, object and document tracking, labeling, speech, among several other related
fields \cite{gu2018recent}. \hl{Paragraph on applications to be completed---with particular attention paid to tree-related efforts.}


\section{Data and Methods}
\subsection{Image data description}
The training dataset consists of 505 images, each having an original size of $4032\times3024$ pixels.   Images were captured over a single field season in Massachusetts, USA, between May and September 2020 to limit any potential influence of changes in tree appearance due to seasonal leaf senescence on image processing.  ESRI ArcMaps was used to randomly distribute sampling sites across the state.  Risk assessments followed the “Level 1” methods outlined in the second edition of the International Society of Arboriculture’s (ISA) Tree Risk Assessment Beast Management Practices (Smiley et al. 2017) and ISA’s Utility Tree Risk Assessment Best Management Practices (Goodfellow 2020).  This method is commonly used to assess trees in the United States.  This level assessment was selected for this study because: (1) individual risk assessments may be prohibitively expensive at higher orders, i.e. Level 2 or Level 3 (Smiley et al. 2017), given the hundreds of thousands of trees utilities must manage across territory areas;  (2) utility rights-of-way (ROW) rights may not allow utility inspectors full access to trees in practical application of higher order risk assessment procedure if the trees are beyond the edge of the ROW (Goodfellow 2020); (3) studies have shown reasonable efficacy of limited basic visual assessment techniques in identifying more severe tree defects (Rooney et al. 2005, Koeser et al. 2016) leading to more severe likelihood of failure ratings.  The four categories of tree failure likelihood are defined as follows: (Smiley et al. 2017 (I have not figured out how to embed citations yet, sorry -Ryan):
\begin{itemize}
\item \textbf{Improbable}: failure unlikely either during normal or extreme weather conditions 
\item \textbf{Possible}: failure expected under extreme weather conditions; but unlikely during normal weather conditions
\item \textbf{Probable}: failure expected under normal weather conditions within a given time frame
\item \textbf{Imminent}: failure has started or is most likely to occur in the near future, even if there is no significant wind or increased load. This is a rare occurrence for a risk assessor to encounter, and may require immediate action to protect people from harm
\end{itemize}
	In this study, only images of trees assigned to the lowest 3 likelihood of failure categories of \textit{improbable, possible, \textit{and} probable} were used due to the rarity of the \textit{imminent} category trees.    
\subsection{Pre-processing}

\subsection{Convolutional neural network}

We can apply cut-out (occluding portions of the image) for improved performance \cite{devries2017improved}. Also, it has been shown that training with lower resolution improves performance on higher resolution test images \cite{touvronfixing}.

\begin{table}[h!]
  \centering
  \begin{tabular}{l l l l l l l }\toprule
    \bf Model & \multicolumn{3}{c}{\bf Training metrics} &\multicolumn{3}{c}{\bf Validation metrics}  \\\midrule
    & Error & Precision & Recall     & Error & Precision & Recall \\
    SafeTree & & & & & & \\
    GoogleNet (InceptionV3) & & & & & & \\
    ResNet50 & & & & & & \\
        VGGNet & & & & & & \\
    AlexNet & & & & & & \\\bottomrule
  \end{tabular}
  \caption{Comparing our model SafeTree with state-of-the-art CNN architectures trained on our data}
  \label{tab:comp}
\end{table}


\section{Results}

\subsection{Sensitivity to training resolution}

\section{Conclusion}

\printbibliography

\appendix


\end{document}

%%% Local Variables:
%%% mode: latex
%%% TeX-master: t
%%% End:
